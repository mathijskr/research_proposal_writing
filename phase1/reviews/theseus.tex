\subsubsection*{Number and title of the proposal}
3. Theseus

\subsubsection*{What is the scholarly, scientific or technological relevance of the problem? Is the problem original, timely, challenging?}
Traditionally, the scientific community assumed that when vulnerabilities are disclosed and patches are released, software systems are quickly modified to improve security. However, the author of the proposal argues that this is not the case. Assuming software systems are not quickly patched is an original problem. The problem is also challenging because it involves dealing with complex human interests and deficiencies; and with governmental structures. Another original problem that the author will face is that currently organizations do not have the risk analysis tools to quantify the risk of whether or not to apply a patch.

\subsubsection*{What are the innovative and original aspects of the proposal? Are the project objectives challenging and scientifically ground-breaking?}
One of the innovative aspects of the proposal is the holistic approach to solving the problem at hand. The author realizes that solely focusing on technological solutions is not enough when dealing with complex governance. Another innovative aspect of the proposal is the assessment of the likelihood of vulnerabilities becoming exploited. Quantifying the exploitability of a vulnerability by automatic exploit generation can be considered scientifically ground-breaking.

\subsubsection*{Is the approach suitable, including a practical work programme? How do you assess the program of work described in the proposal (realistic, feasible, ...)?}
The project is divided into five work packages. Each work package is being lead by the group which experience best fits the work package. Each work package is further divided into individual projects. This organization allows individual projects to fail without making the entire project a failure. However, the author does not present a detailed risk analysis of what could go wrong and how eventual contingencies could be mitigated. I am therefore not fully convinced of the feasibility of the project, especially considering the project will deal with different types of organizations: companies, research groups, and governmental institutions.

\subsubsection*{Is the proposal well-written and are the project’s objectives clearly worded?}
The proposal reads pleasantly because of the organization into sections and paragraphs. Additionally, the work packages are all presented in the same layout and in a way that they can easily be digested without necessarily reading the context. However, a specific issue that the proposal has is a lacking methodology for work package five. It is insufficiently clear how the author plans to quantify the results of the project and compare it to the status quo, which makes it difficult to assess the success of the project.

\subsubsection*{What is your opinion on the scientific, societal and/or economic impact of the proposed research? Are the expected results of the research relevant for solving a societal/economic/cultural/ technical or policy-related challenge?}
The scientific impact of the project is limited to the results of quantifying the exploitability of vulnerabilities using automatic exploit generation. However, the project has the potential to have a bigger impact on solving societal, economical, and policy-related problems. If the project completes successfully it may result in less vulnerable computer systems in industry and governments. This will have a positive impact on privacy, safety of critical systems and will reduce economical losses experienced by companies.

\subsubsection*{How do you assess the entire application? Please give the final scoring for your assessment, on a 1(low) to 10 (high) scale.}
8.

\subsubsection*{Could you please summarise or briefly comment (point by point) on the strengths and weaknesses of the proposed research? [Note that this question is about the proposed research and not about the writing style of the proposal!]}
\paragraph{Strengths}
\begin{itemize}
    \item Originality of assuming vulnerabilities are not quickly patched.
    \item Holistic approach.
    \item High societal, economical, and policy-related impact potential.
    \item Developing a method for quantifying exploitability of a vulnerability.
    \item Utilizing knowledge of diverse work groups.
\end{itemize}

\paragraph{Weaknesses}
\begin{itemize}
    \item Success of project is insufficiently quantifiable.
    \item No detailed risk analysis nor backup plan at each stage.
\end{itemize}
