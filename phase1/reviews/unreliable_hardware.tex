\subsection*{Number and title of the proposal}
1. How I Learned to Stop Worrying and Love Unreliable Hardware

\subsection*{What is the scholarly, scientific or technological relevance of the problem? Is the problem original, timely, challenging?}
- Context of increasing hardware vulnerabilities.
- Higher transistor density -> decreased reliability.

\subsection*{What are the innovative and original aspects of the proposal? Are the project objectives challenging and scientifically ground-breaking?}
- Generic defense against all hardware vulnerabilities.
- "Bandage", real solution would be fixing hardware.
- Not foundational research, big practical and engineering part.

\subsection*{Is the approach suitable, including a practical work programme? How do you assess the program of work described in the proposal (realistic, feasible, ...)?}
- Three projects.
- Reverse engineering all existing and future chip row mappings is not realistic. More feasible to come up with a protocol that hw vendors can implement.
- Other two projects are feasible.

\subsection*{Is the proposal well-written and are the project’s objectives clearly worded?}
- Duplication between challenges and work packages.
- Figures support the explanation of presented concepts.

\subsection*{What is your opinion on the scientific, societal and/or economic impact of the proposed research? Are the expected results of the research relevant for solving a societal/economic/cultural/ technical or policy-related challenge?}
- Prevents real world attacks.
- Performance impact.
- Enables new research, e.g. the rowhammer emulator.

\subsection*{How do you assess the entire application? Please give the final scoring for your assessment, on a 1(low) to 10 (high) scale.}

\subsection*{Could you please summarise or briefly comment (point by point) on the strengths and weaknesses of the proposed research? [Note that this question is about the proposed research and not about the writing style of the proposal!]}
